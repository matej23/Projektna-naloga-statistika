\documentclass{article}
\usepackage[utf8]{inputenc}
\usepackage[T1]{fontenc} 
\usepackage[slovene]{babel} 

\usepackage{enumitem} 
\usepackage{hyperref}
\usepackage{amsmath}
\usepackage{amsthm} 
\usepackage{amssymb}
\usepackage{lmodern}
\usepackage{amsfonts} 
\usepackage{mathtools}
\usepackage{graphicx}
\usepackage{float}

\DeclareMathOperator{\EX}{\mathbb{E}}


\begin{document}

\title{Simetrična diskretna verižnica z liho členki\\
    \large Projekt pri predmetu Matematično modeliranje
}
\author{
    Matej Novoselec\\
}
\date{30.\ junij 2023}

\maketitle

\section{Naloga 2}
Vredno je zapisati nekoliko bolj matematično interpretacijo problema/navodila. 
Podatki pripadajo trem različnim eksperimentom, ki bodo določali končne vrednosti in rezultate, a so problemi, ki jih podajajo, teoretično iste narave. 
Za podan problem sedaj razvijemo teoretični pristop in se lotimo reševanja podnaloge a) in b). 
\newline
Imamo $n$ neodvisnih, enako porazdeljenih slučajnih spremenljivk, označimo jih z $R_1,~R_2,~\dots,~R_n$. 
Porazdeljene naj bodo Rayleighovo, t.j. z gostoto:
$$
f_{R_i}(r_{i} \mid \theta)=\left\{\begin{array}{cl}
\frac{r_i}{\theta^{2}} \exp \left(-\frac{r_{i}^{2}}{2 \theta^{2}}\right) & ;~~r>0 \\
0 & ;~~\text{sicer }
\end{array}\right. .
$$
Zaradi predpostavljene neodvisnosti, je potem $(R_1,~R_2,~\dots,~R_n)$ porazdeljen z gostoto:
$$
    \prod_{i=1}^{n}{f_{R_i}(r_{i} \mid \theta)} = 
    \bigg(\frac{1}{\theta^{2n}}\prod_{i=1}^{n}{r_i}\bigg) \exp\bigg(-\frac{1}{2 \theta^{2}} \sum_{i=1}^{n}{r_i^2}\bigg);~~r_i>0.
$$
Lotimo se podnaloge a). Velja:
$$
L(\theta \mid (r_1,~r_2,~\dots,~r_n)) = \prod_{i=1}^{n}{L_i(\theta \mid r_i)} = \prod_{i=1}^{n}{f_{R_i}(r_i \mid \theta)}
$$
in
$$
l(\theta \mid (r_1,~r_2,~\dots,~r_n)) = \ln L(\theta \mid (r_1,~r_2,~\dots,~r_n)) = \sum_{i=1}^{n}{\ln \big(f_{R_i}(r_i \mid \theta)\big)}, 
$$
ter zato:
$$
l(\theta \mid (r_1,~r_2,~\dots,~r_n)) = -2n \ln(\theta) + \sum_{i=1}^{n}{\ln(r_i)} - \frac{1}{2\theta^2} \sum_{i=1}^{n}{r_i^2}~.
$$
Iščemo cenilko za $\theta$ po metodi največjega verjetja, zato si ogledamo enakost:
$$
0 = \frac{\partial l(\theta \mid (r_1,~\dots,~r_n))}{\partial \theta} = - \frac{2n}{\theta} + \frac{1}{\theta^3}\sum_{i=1}^{n}{r_i^2}.
$$
Za cenilko po metodi največjega verjetja tako dobimo:
$$
\hat{\theta} = \sqrt{\frac{1}{2n}{\sum_{i=1}^{n}{r_i^2}}}~.
$$
Za rešitev podnaloge b) si oglejmo pričakovano vrednost (Rayleighove) slučajne spremenljivke $R$:
$$
\EX(R) = \int_{0}^{\infty}{r~f(r \mid \theta)~dr}= \int_{0}^{\infty}{\frac{r^2}{\theta^{2}}~\exp\Big(-\frac{r^{2}}{2 \theta^{2}}\Big)~dr}.
$$
Uvedemo $\tau = \frac{r^{2}}{2 \theta^{2}}$ in dobimo
$$
\EX(R) = \int_{0}^{\infty}{\sqrt{2 \tau}~\theta ~e^{-\tau}~ d \tau} = \theta \sqrt{2}\int_{0}^{\infty}{\tau^{1/2}~e^{-\tau}~ d \tau} = \theta\sqrt{2}~\Gamma(3/2) = \theta \sqrt{\frac{\pi}{2}}.
$$
Cenilka po metodi momentov je zato:
$$
\hat{\theta} = \overline{R} \sqrt{\frac{2}{\pi}}~.
$$






\bibliographystyle{siam}
\begin{thebibliography}{9}
    \bibitem{clanek}
        E.~Zakrajšek, \emph{Verižnica}, [ogled 30.~6.~2023], dostopno na \url{https://ucilnica.fmf.uni-lj.si/pluginfile.php/8283/mod_resource/conte4/predavanja/veriznica/veriznica.pdf}.
    \bibitem{zapiski}
        E. Žagar \emph{Zapiski predavanj 22.3.2021 - Problem diskretne verižnice} [ogled 29.~6.~2023], dostopno na \url{https://ucilnica.fmf.uni-lj.si/pluginfile.php/100122/mod_resource/content/1/mm_uni_22_3_21.pdf}.
\end{thebibliography}

\end{document}